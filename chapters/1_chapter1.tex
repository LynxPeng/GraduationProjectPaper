\chapter{绪论}

\section{研究背景及意义}

\subsection{选题背景}

当今计算领域对操作系统的性能、安全性和可靠性要求日益严苛,微内核技术因其固有的优势而备受关注。其中,seL4 微内核作为目前最先进的微内核之一,已被广泛应用于安全关键领域。然而,seL4 的同步系统调用和 IPC 会产生大量的特权级切换,且无法充分利用多核的性能。虽然微内核对异步通知有一定的支持,但仍需要内核进行转发,其中的特权级切换开销在某些平台和场景下将造成不可忽视的开销。

为了进一步提高系统性能和效率, ReL4 项目使用 rust 语言重写的 seL4 微内核,基于用户态中断技术改造 seL4 的通知机制,设计了无需陷入内核的异步系统调用和异步 IPC 框架,在提升用户态并发度的同时,减少特权级的切换次数。而 ReL4 项目中,异步 IPC 的引⼊造成了额外的运⾏时开销,导致异步 IPC 在低并发的场景下性能显著低于同步 IPC。

\subsection{研究目的}

本研究旨在针对 ReL4 项目中异步进程间通信的性能瓶颈问题,提出并实施了一种创新性的解决方案。该方案通过利用硬件资源,部分替代传统的异步调度器功能,对异步运行时进行硬件层面的加速,以期显著提升异步 IPC 的性能表现。

\subsection{研究意义}

本研究旨在提高了 ReL4 项目中异步进程间通信的性能,此研究成果可直接迁移至实际操作系统开发,以增强系统运行效率,降低资源消耗,进而提升用户交互体验。

在安全性及可靠性增强方面,本研究通过降低特权级切换频率,不仅优化了操作系统性能,亦提升了系统的安全性和可靠性。此优化对于安全关键领域,如航空航天、军事、医疗等,具有尤为重要的意义,有助于确保这些领域信息系统的安全稳定。

在技术应用与产业推动层面,本研究为微内核操作系统开发者提供了切实可行的技术路线,推动了微内核技术在多核处理器环境下的广泛应用。同时,本研究对于硬件加速技术的应用具有典范作用,有助于促进相关产业的技术进步和创新。

此外,本研究促进了操作系统与硬件设计、并发编程等领域的深度融合,为跨学科研究提供了新的研究路径和方法论。这对于促进计算机科学与其他工程学科的技术融合,具有重要的学术价值和实践意义。

\section{国内外研究现状}

目前,国内在硬件调度器领域已经有了一些进展。如关沫张晓宇采用 VHDL 语言设计了适
用于硬件化的实时操作系统调度器,基于 FPGA 使用组合电路和时序电路完成了系统内核调度
器的搭建。
国外学者 Y.Klimiankou 提出了一种提出了 Micro-CLK,一种基于微内核的多服务器操作
系统设计,其核心思想是将进程间通信从内核中移除,从而提高效率并简化内核设计。进程间
通信的优化成为微内核性能优化的重点。
目前的研究工作虽然在不同领域取得了进展,但没有将硬件调度器与异步的进程间通信与
异步的系统调用结合的实例。

\section{研究内容和研究方法}

本研究将围绕以下几个核心环节展开:
首先,本研究将对 Rel4 内核中的用户态异步运行时机制以及 TAIC 硬件调度器的设计理
念与具体实现进行详尽剖析,将深入挖掘其工作原理、性能特点以及接口设计,为后续的优化
与改进提供理论基础。
其次,将着手进行硬件适配的异步运行时的体系结构设计与编码,主要包含以下工作:
a. 将内核中异步运行时的队列改为 taic 实现,以适应异步系统调用。
b. 将异步系统调用的库函数实现由系统调用改为读写 taic。
c. 将异步系统调用回复时的用户态中断改为读写 taic,唤醒用户态对应的协程。
接下来,将在 Qemu 模拟器和 FPGA 硬件环境两种不同的平台上进行仿真与测试,以确保
我们的异步运行时能够在多种场景下稳定运行。主要测试内容包括:
a. 对改进前后的异步 ipc 和异步系统调用在不同并发度下进行测试。
b. 对改进前后的异步 ipc 和异步系统调用低并发下不同环节的时间消耗进行测试。
最后,我将对仿真与测试的结果进行全面评估与分析,针对发现的问题进行调优。通过这
一系列的优化措施,旨在提升异步系统调用的性能,使其更好地适应各种应用场景。

\section{论文结构安排}

\section{本章小结}

