\chapter{绪论}

\section{研究背景及意义}

在操作系统研究领域,微内核架构因其模块化、可扩展性和隔离性,成为未来资源受限边缘计算和分布式系统的重要基础架构选择\cite{nandy2024resource}。其中,seL4作为全球首个实现形式化验证的微内核,在正确性与安全性保障方面达到了前所未有的水平,是安全和关键任务系统的理想基础\cite{heiser2024lionsos}。
然而,seL4采用同步IPC作为其核心进程间通信机制,该机制在高负载场景中会带来较严重的性能瓶颈\cite{mergendahl2022thundering}。

为提升操作系统的并发处理能力,异步化改造成为学术界研究的重点方向。其中,异步系统调用具备非阻塞、低延迟等优势,在多核系统下对高并发任务的处理具有天然的优势。在此背景下,ReL4项目\cite{rel4_kernel}对seL4系统进行异步化改造,基于用户态中断和异步运行时实现了高性能的异步进程间通信与异步系统调用。实验结果表明,Rel4对高并发场景下的系统吞吐率具有很明显的提升。

然而,Rel4系统在处理低并发异步系统调用时平均耗时较高,甚至不如原有的同步调用。造成此现象的主要原因有两方面。一方面,Rel4中的异步调度器基于软件逻辑实现,当系统负载较低时,异步运行时的引入所造成的额外开销难以通过并发均摊。另一方面,该系统中异步系统调用的过程涉及多个协程调度与上下文切换。低并发所带来的调度频率提升加剧了上下文切换对系统整体性能的负面影响,进而降低了异步系统调用在低并发场景下的平均耗时。

该问题虽然可以通过同步IPC解决,但多种IPC机制的引入违背了微内核设计中的“最小化”原则。Heiser 和 Elphinstone 指出,引入同步与异步两种通信方式会破坏微内核的纯粹性,尤其同步IPC在多核环境下会强行串行化客户端与服务器的执行流程,限制了并发性,因此基于异步通知的通信协议会是未来的最优选择\cite{heiser2013from}。综上所述,在以异步IPC为主导的Rel4系统中,针对低并发场景进行性能优化至关重要。

本文提出一种基于硬件控制器的优化方案,针对ReL4系统中的性能瓶颈问题进行优化。该方案利用TAIC(任务感知中断控制器)对ReL4系统中的异步路径进行优化重构,实现了无需陷入内核的异步系统调用。该方案中,硬件资源取代了部分原有异步运行时的工作,在单核情况下优化了任务调度效率,在多核情况下彻底消除了内核陷入所带来的额外开销,显著降低了异步系统调用在低并发场景下的平均用时。

\section{国内外研究现状}

\subsection{微内核与异步通信机制}

微内核架构因其“最小化”的设计理念被广泛应用于高可信度场景,然而,越来越多的用户态系统服务使得进程间通信成为系统性能的主要瓶颈。随着多核处理器广泛部署,同步IPC机制的性能缺陷日益明显。国外学者指出,在任务切换频繁场景中,同步IPC带来的内核陷入与上下文切换频次成正比,严重制约了系统吞吐率 \cite{reichmann2015ipcperformance}。

为了解决这一问题,研究者们开始尝试将异步通信机制引入微内核中。
国外学者Aigner等人在对微内核系统的通信方式分析时指出异步服务器模型在系统响应速度和模块隔离性方面优于传统同步方式 \cite{2011Communication}。此外,OKL4已完全放弃了同步IPC,改为使用虚拟中断和异步通信机制\cite{varanasi2010okl4}。
而后,国内学者在OKL4平台上设计了一种基于事件通道和共享内存的数据通道机制,实现了虚拟机间异步数据传输,显著提升嵌入式系统的IPC效率\cite{wang2017efficient}。

这些工作表明,异步IPC已逐步取代同步IPC成为多核微内核通信的主要方向。

\subsection{无需陷入内核的异步路径优化}

虽然异步机制的设计通过充分利用阻塞等待时间提升了系统的吞吐率,但目前,大多数异步通信的设计仍需上下文切换完成进程调度。而上下文切换所造成的存储恢复开销和TLB污染开销使得异步通信的性能存在局限性。

Micro-CLK是由国外学者提出的一种微内核架构。在该架构中,进程间事件和状态调度完全由用户态管理,主张通过完全避免使延迟优化降低到最小\cite{Klimiankou2021microclk}。

Skybridge系统\cite{2019SkyBridge}通过虚拟地址空间映射,使进程可以直接通过IPC调用目标进程的函数。该系统不修改硬件结构的前提下,通过虚拟指令实现了,完成了地址空间的转换。其实验表明,与seL4标准IPC路径相比,Skybridge通信延迟降低65\%,吞吐能力提升近5倍。但该方法只适用于虚拟化环境中。

这些研究表明,减少上下文切换对提升异步通信的延迟效果显著且至关重要。

\subsection{基于硬件加速的进程间通信}

为减少上下文切换,提升异步系统性能。学者普遍通过硬件辅助以减少上下文切换。

国内学者提出一种硬件辅助操作系统原语,XPC\cite{XPC2019}。该原语通过硬件辅助提供了无需陷入的跨进程同步调用。该方法与当前主流的虚拟内存模型兼容,并可集成到现有的微内核操作系统中。然而,该方法仅在FPGA上有所实现,缺少统一的硬件标准,且仅支持同步进程间调用,因此适用范围有限。

ReL4操作系统中使用risc-v提供的用户态中断能力绕过内核,实现无需陷入的进程间通信。然而,该此方法的开销较高,在低并发场景下具有明显的性能缺陷。此外,由于内核无法接收用户态中断,基于该方法实现的异步系统调用仍需进行一次陷入来调度内核中的线程。因此,该系统缺少对低并发场景及异步系统调用的有效优化。

任务感知中断控制器(TAIC)\cite{taic2024}是由国内学者研发的硬件。该控制器通过MMIO寄存器接口访问。此外,该控制器提供的队列具有任务调度能力,能够基于中断触发,将任务由硬件抽象的阻塞队列加入就绪队列。通过该硬件触发中断,可彻底避免上下文切换所承担的开销,并且不会有地址空间切换所带来的安全问题,具有很高的可适用性。

\section{主要研究内容}

本文对Rel4系统在低并发场景下异步系统调用性能较低的原因进行分析,并基于TAIC对提出一种解决方案。基于该解决方案,完成TAIC接口层的实现,对原有的缓冲区结构进行优化,优化异步运行时中部分处理逻辑并重构了异步系统调用路径。为验证实现正确性和优化效果,本设计在qemu模拟平台和FPGA平台对实现结果进行功能性测试与性能测试,对实验结果进行总结分析并给出结论。

本研究的关键点在于如何设计硬件资源的分配和如何选择中断注册机制。

硬件资源的有限性导致无法将所有异步任务的唤醒均交由硬件处理。本研究为解决硬件资源有限的问题,设计一种资源分配与映射机制,并在接口层对该机制进行了封装。

硬件调度器提供了两种中断注册机制并各有优缺点。为验证不同中断注册机制对系统性能的影响,本研究实现了两种中断注册逻辑并设计对比实验对其进行验证。


\section{论文结构安排}

本文的主要结构安排如下:

第一章为本文的绪论部分,该章节对本设计的研究背景及意义,国内外发展现状,研究主要内容进行了说明。

第二章对本设计的相关工作进行了阐述,介绍了rust异步语义及其异步运行时实现,对Rel4异步系统调用的设计和TAIC硬件调度器的工作流程进行了详细说明。

第三章主要讲述优化后的系统设计。该章节首先对Rel4系统中的性能瓶颈进行分析,然后介绍了改造后的整体系统架构,并从硬件资源分配、缓冲区结构优化、异步事件注册三个方面详述设计细节。

第四章阐述了本设计的实现细节,包括各层级的具体实现及关键性算法。

第五章阐述了对本设计的实验评估。实验从功能性测试以及性能测试两方面对优化后的性能进行验证,并针对两种中断注册机制进行对比实验。本文对实验结果进行分析并得出结论。

\section{本章小结}

本章作为绪论,介绍了微内核架构及其代表系统seL4的研究背景,重点阐述了异步化改造项目Rel4在提升高并发场景下系统吞吐率方面的优势及其在低并发条件下存在的性能瓶颈。分析指出,Rel4异步调度器的软实现导致低负载下额外开销难以摊销,且频繁的协程调度与上下文切换加剧了性能下降,亟需优化方案。结合国内外基于硬件调度器优化系统性能的相关研究,本文提出基于TAIC硬件调度器的优化设计,通过硬件资源替代部分异步运行时功能,减少上下文切换开销,实现无需陷入内核的异步系统调用。章节还简要介绍了研究的主要内容和关键技术,包括硬件资源分配、中断注册机制设计及接口层实现,并对全文结构进行了说明,为后续章节的深入展开奠定了基础。
