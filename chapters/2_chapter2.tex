\chapter{相关工作论述}

\section{Rust并发模型}

Rust是一门强调并发性与安全性的系统编程语言,其凭借所有权系统、生命周期及智能指针等一系列机制可以在编译期保障系统的内存安全问题。标准Rust语言的性能与C或C++类似,其高效的并发模型和零成本抽象模式为异步编程提供了良好的环境。目前,Rust语言已被广泛应用于操作系统内核的开发中,如Redox OS\cite{redoxdocs}、Tock OS\cite{Tockos}。此外,Linux内核也开始引入Rust语言,用于编写部分新的内核驱动和模块,目的是提高安全性和可靠性\cite{linux-kernel-rust-docs}。

ReL4系统中使用Rust作为系统编程语言,并基于Rust的并发模型构建异步运行时,以实现内核与用户线程中的异步任务调度。

\subsection{Rust的异步编程模型}

Rust语言采用async模型作为其异步编程模型。该模型基于协程和事件驱动,其核心是通过async/await语法将异步逻辑转化为类似同步代码的结构。具体而言,该模型中被声明为异步的async函数在编译期会被转换为一种实现了future接口的状态机。future中的异步计算在声明时不会被执行,当运行时调用poll方法轮询future时,状态机才会逐步执行。基于该模式的async内部实现没有任何性能损耗,因此Rust具有优秀的异步编程性能。Rust通过该模型避免了线程阻塞和“回调地狱”,使异步任务的创建和切换开销显著低于多线程模式。

future代表一个尚未完成的异步计算,它可以被poll方法轮询执行,并在未来某个时间点返回一个计算结果。Future trait的简要定义如下:

Poll枚举用于表示Future的当前状态。它有两个可能的枚举值Ready(T),和Pending。Ready(T)表示Future已经完成,并返回一个值T;Pending表示Future尚未完成,需要继续轮询。

poll方法用于尝试调度future中的异步计算继续执行。该方法返回一个Poll枚举表示Future的完成状态,poll方法的调用者以此判断Future是否完成。

Pin用于固定状态数据在内存中的地址。由于异步函数被编译为状态机执行,异步函数中的局部变量等字段也会被编译成状态字段。异步函数的执行中可以会改变状态机在内存中的位置,一旦异步函数中涉及临时变量的引用,则会不可避免的出现悬空指针。Pin<T>类型则保证被包裹的数据在未显式解除Pin的情况下不能移动。通过Pin<\&mut Future>的方式传入poll方法,可以确保状态机在整个生命周期中位置固定。

async和await是Rust中的两个关键字,是异步语义中的核心语法糖,它们提供了更简洁的编写异步函数的方式。
async用于声明一个异步函数。编译器会将async函数转换为实现了Future trait的状态机。
await用于阻塞当前的异步函数,并等待一个Future完成。await的阻塞并不会影响其他future的执行。当future完成后,await会解构future的返回值。


\subsection{Rust的异步运行时}

Rust的异步运行时围绕 Executor和Reactor设计。Executor是一个调度器,负责管理和调度异步任务的执行。Reactor是一个事件循环,负责等待IO事件的发生,并将事件通知到对应的异步任务。

图\ref{asyncruntime}给出了异步运行时中Executor和Reactor的设计框架:

\begin{figure}[htbp]
    \centering
    \includesvg[inkscapelatex=false,width=0.85\textwidth]{images/async_runtime}
    \caption{异步运行时设计框架}\label{asyncruntime}
\end{figure}

Executor是Rust异步编程模型中的一个重要组件,它负责管理future的执行,并提供运行时环境。
Executor会提供异步任务所需的资源并使用一定的调度策略决定哪个异步任务被优先执行。在进行异步任务时,Executor会轮询future中的poll方法,直到所有future均执行完毕。
Rust标准库中并没有提供默认的Executor实现,开发者可使用Rust社区中的异步运行时库。

Task表示一个正在执行的异步任务。当一个future被提交给执行器并被调度执行时,执行器会为该Future创建一个Task,并将其加入任务队列。Task负责管理 Future 的生命周期,并在Future完成时通知执行器。执行器通过管理task来管理异步任务的执行。

Waker是一个用于唤醒任务的句柄,每个异步任务的执行上下文中有一个Waker对象。Waker对象实现了wake trait,提供了wake方法,可用于唤醒已经阻塞的异步任务。具体而言,当future因等待其他事件而阻塞后,执行器将不再尝试轮询该future,而是将其状态设置为Pending。当异步任务因IO完成或其他事件准备就绪后,Waker会被调用以通知执行器唤醒该任务。唤醒后的任务会重新进入执行队列,等待执行器的调度。

\section{ReL4系统设计}

ReL4是基于Rust语言实现的异步微内核,其遵循seL4的微内核架构理念并对其进行创新性优化。
在与seL4的系统调用保持兼容的基础上,ReL4对IPC机制进行了异步化改造,避免了IPC过程中的特权级切换,实现了高性能的异步进程间通信。

ReL4可复用seL4的构建工具链,仅依赖其文件和部分配置项。在构建时,ReL4内核会被编译为动态链接库,通过seL4构建工具链与预编译的seL4头文件进行链接,形成可执行的内核镜像。ReL4的用户态根任务使用Rust语言基于rust-sel4开源项目编写。

为实现高性能的异步进程间通信和异步系统调用,ReL4在内核和用户线程库中实现了异步运行时。该运行时基于Rust的异步编程模型设计,提供了异步任务的调度和管理功能。

\subsection{能力机制与主要内核对象}

ReL4采用与seL4相似的能力安全模型作为其访问控制体系。在该模型下,可被用户态线程使用的系统资源被抽象为内核对象。线程访问对应的内核对象需要具有相应的能力并提供能力句柄。ReL4中的主要内核对象包括:TCB、CSpace、VSpace、Notification、Endpoint、Untyped和Frame。


TCB(Thread Control Block)是用于管理线程的核心内核对象,其中存储线程的控制信息,其中包含线程的寄存器快照以及上下文信息用于线程调度。TCB还负责维护与线程调度相关的字段,例如线程的优先级、线程状态等。此外,TCB中存有一系列能力槽,用于存储线程所持有的能力,包括线程的能力空间、虚拟内存空间、通知对象、端点对象和回复能力等。


CSpace(Capability Space)是线程的能力空间,是管理线程可访问资源的能力集合。在实现上,CSpace是层级能力结点(CNode)组成的树形结构。而CNode是一个固定大小的能力槽数组,其中每个槽位可以存储一个能力(capability)。CSpace的层级结构支持能力的派生、重定向和精细化权限控制。线程的能力可由父线程派生,或通过IPC传递给其他线程。在上下文切换时,线程可通过TCB中的指针绑定到自己CSpace的根CNode,从而获得其能力空间视图。


Notification和Endpoint分别是ReL4通知机制中的通知对象和端点对象。Notification对象通常与线程或Endpoint对象相关联,用于传递类似中断、资源就绪等非阻塞事件。Endpoint作为通知的作用端点,可被多个线程用于发送(send)或接收(receive)消息,通信过程支持同步阻塞与非阻塞模式。关于ReL4中的通知机制,将在本文\ref{sec:notification}小节中做更详细的陈述。


VSpace(Virtual Address Space)用于描述线程虚拟地址空间,其负责维护虚拟地址到物理内存之间的映射关系。每个进程在运行时拥有独立的VSpace,以实现内存隔离和访问控制。VSpace由顶层页目录和多级页表构成,在具体实现上依赖底层硬件架构的MMU机制。在创建新进程时,必须显式分配一个VSpace并绑定至该线程的TCB中,确保其拥有合法的地址空间。VSpace还支持地址空间的动态管理,例如映射、撤销或更新特定的虚拟地址区域。


Untyped对象是表示所有可重分配内存资源。它代表一段尚未类型化的物理内存区域,是系统能力分配与内存管理的基本单位。内核启动后,所有未被使用的物理内存区域都被封装为Untyped对象,由根任务(root task)通过能力机制管理。在系统运行过程中,用户线程可通过Retype操作将Untyped对象转化为其他具体类型的内核对象(如TCB、Frame、CNode等)。Retype操作必须满足对齐和大小等约束条件,且一段Untyped内存一旦被分配则不能重复使用,除非通过回收机制恢复。通过能力控制,系统能够确保Untyped对象不会被任意访问或滥用,从而实现安全、细粒度的资源隔离和分配策略。


Frame对象代表ReL4中的基本物理页框资源,是虚拟内存管理中的核心单元。每个Frame对象对应一页固定大小的物理内存,可被映射(map)至VSpace中的特定虚拟地址。线程只能映射其拥有能力的Frame对象,从而防止非法地址访问。ReL4将部分内存资源管理的工作下放至用户态,支持Frame能力的复制与派生,允许多个线程共享同一物理页。此外,Frame对象还可以与设备内存映射,从而支持用户级设备驱动访问硬件资源。


\subsection{通知机制与用户态中断}\label{sec:notification}

ReL4的通知机制与seL4类似,用与线程间的异步事件通知。其核心设计目标是在保持系统微内核结构精简性的同时,实现低延迟、低开销的异步事件交互机制。

\subsubsection{seL4的通知机制}

在seL4中,通知对象(Notification)可以与一个或多个线程进行关联。每个线程的TCB中包含一个可选的Notification挂载字段,用于绑定线程与某个Notification对象。成功绑定的Notification的线程可以通过系统调用接口(Wait、Signal等)与该Notification对象进行交互。

该机制允许通知对象直接作用于线程调度,从而在信号到达时以最小的延迟唤醒目标线程,它为线程提供了一种在不占用CPU的情况下等待事件发生的方法,从而避免频繁轮询导致的资源浪费。

\subsubsection{ReL4的通知机制}

ReL4系统中通知按特权级可分为以下四类

\begin{enumerate}
    \item 用户态之间的通信
    \item 内核向用户态的通知
    \item 用户态向内核的通知
    \item 内核态之间的通知
\end{enumerate}

ReL4系统为减少通知过程中的特权级切换,对seL4的通知机制进行了优化。对于第一种和第二中情况,ReL4基于用户态中断对原有的通知处理逻辑进行了重构。对于第三种情况,ReL4采用系统调用的方式将消息发送至内核。第四种情况本身不涉及上下文的切换。

\subsubsection{用户态中断}

ReL4 系统引入了用户态中断机制(U-notification),通过实现控制流和数据流的分离来降低通信过程中的特权级切换开销。该机制依托一个特殊的用户态中断控制器(UINTC),在通知通信过程中,控制流由用户主动向内核注册所需资源,而数据流则直接通过硬件传递,不再经由内核。

控制流包括发送方和接收方的注册两个部分。

接收方注册:用户线程首先创建Notification对象,然后通过TCB\_Bind接口将其绑定到线程TCB。随后调用 UintrRegisterReceiver系统调用,注册中断向量表并向 UINTC申请接收状态表项。

发送方注册:发送方通过能力派生机制获得对 Notification对象的访问权限。在第一次执行发送操作时,若该能力尚未包含Sender ID,则会触发UintrRegisterSender系统调用以完成注册并分配Sender ID。

注册完成后,发送方可直接执行 uipi\_send 指令发送用户态中断。

\subsection{异步运行时设计}

由于ReL4内核中去除了同步IPC的支持,ReL4系统在用户态提供了异步运行时,该运行时提供了用于异步通信的共享缓冲区和协程调度器。

为实现轻量化的异步通信,ReL4选用协程作为调度的基本单位。协程调度器整体实现框架如图\ref{Executor}所示。

\begin{figure}[htbp]
    \centering
    \includesvg[inkscapelatex=false,width=\textwidth]{images/executor}
    \caption{协程调度器框架图}\label{Executor}
\end{figure}


异步运行时将协程分为worker协程和dispatcher协程。worker协程用于处理异步请求,dispatcher协程用于处理中断信号并执行协程唤醒。为了解决不同协程之间的处理顺序问题,协程调度器引入了优先级调度机制,并在内部维护了一个多级优先级队列。协程在创建时可被赋予一定的调度优先级。通过设置协程的优先级可以影响调度器的调度策略实现针对性的性能调优。

异步运行时的共享缓冲区是一个环形队列结构,队列中可存放一定数量的异步通信消息(IPC\_Item)。IPC\_Item由协程id(cid),消息信息(msg\_info)和扩展消息信息(extended\_msg组成)。队列通过原子变量实现跨线程的安全读写。为减少不必要的用户态中断收发,缓冲区内部加入了两个布尔类型的原子变量用于标记协程的执行状态。

\subsection{异步IPC流程}

本章节将对ReL4系统中基于用户态终端的异步IPC流程进行介绍。以最常见的call流程为例,在异步通信典型场景中,双方分别为客户端和服务端。客户端通过call请求向服务端发送ipc通信,并等待服务端的回复。服务端接收到请求后,进行处理并通过reply回复客户端。基于用户态中断的异步IPC流程如图\ref{asyncipc}所示。

\begin{figure}[htbp]
    \centering
    \includesvg[inkscapelatex=false,width=\textwidth]{images/async_ipc}
    \caption{基于用户态中断的异步IPC}\label{asyncipc}
\end{figure}


在用户线程的初始化阶段,系统会按照\ref{sec:notification}章节中描述的用户态中断流程对客户端和服务端分别进行注册并分配共享缓冲区用于异步通信。

当客户端需要进行异步IPC时,会首先依靠异步运行时创建一个异步协程,在协程内部会调用异步IPC接口。在发起异步IPC请求的过程中,协程会构造IPC请求消息并写入异步IPC缓冲区,根据缓冲区中的状态变量判断是否需要唤醒服务端的处理协程。若服务端的处理协程处于阻塞状态,则对服务端线程会发起用户态中断,然后阻塞自身等待回复。

当服务端线程接收到用户态中断后,会进入中断处理函数。在该函数中通过异步运行时的wake方法唤醒服务端的分发协程(dispatcher)。分发协程会读取缓冲区中的请求消息,唤醒服务端的目标协程(也就是处理IPC消息的协程)。

该协程处理完IPC请求后,会使用同样的方法对客户端的协程进行回复。客户端也使用相同的方法唤醒原协程并读取缓冲区中的回复消息,完成整个异步IPC的call请求过程。

异步IPC中的send请求与call请求类似,区别在于send请求后客户端不需要阻塞等待服务端回复。

ReL4系统中的异步系统调用可以看作是特殊的异步IPC请求。其流程与异步IPC类似,区别在于异步系统调用的接收方为内核中的处理协程。由于用户态中断无法发送至内核,因此ReL4系统中的异步系统调用依靠系统调用辅助实现。该部分在\ref{sec:syscallanalysis}章节中进行详细介绍。

\section{TAIC任务感知中断控制器}\label{sec:taic}

TAIC是一种任务感知中断控制器。与传统的中断控制器不同,TAIC能够根据任务的状态和优先级动态调整中断的处理方式。它通过硬件调度器来管理中断的分发和处理,从而提高系统的响应速度和资源利用率。

TAIC的创新性体现在两个方面。首先,在状态管理机制上,TAIC 彻底打破了传统中断控制器依赖软件查表的模式,将任务状态信息的维护和访问职责下沉到硬件。每个任务在系统初始化阶段便会将其状态信息注册至硬件中,后续中断发送过程中无需访问内存。该方法显著减少了处理器访问内存所带来的延迟。

其次,TAIC将任务调度逻辑集成于中断控制器内部,这种设计突破了传统中断处理流程的限制。在传统系统中,中断服务例程通常仅负责处理中断源的基本事件通知,将任务唤醒交由软件处理,这一过程涉及上下文的切换,带来较高的性能开销。而在TAIC中,中断注册阶段就将相关处理任务与具体中断源进行绑定。当中断信号到达时,TAIC会立即根据硬件中的任务映射信息,根将对应任务加入硬件就绪队列。

\subsection{系统整体架构}

\begin{figure}[htbp]
    \centering
    \includesvg[inkscapelatex=false,width=\textwidth]{images/taic_arch.svg}
    \caption{TAIC系统架构图}\label{taicarch}
\end{figure}

图\ref{taicarch}给出了基于软硬协同的任务调度和中断响应系统整体架构\ref{taic2024}。TAIC由就绪队列、发送能力表、接收能力表和中断处理模块组成。就绪队列根据任务标识中与调度相关的属性维护任务标识的顺序,且直接支持负载均衡功能;发送能力表中记录了可以发送任务通知的发送方标识,而接收能力表中则记录了可以接收中断事件和任务通知的接收方任务标识。外部设备的中断信号连接到TAIC的中断处理模块中,软件通过总线与TAIC交互,通过软硬件交互接口使用TAIC的调度功能以及注册与中断事件和任务通知相关的能力,中断处理模块在收到中断或者任务通知信号时,会根据中断号或者通道号来检查对应的发送方能力表或者接收能力表,从而完成中断处理和任务通知。

\subsection{驱动设计及使用方法}\label{sec:taicdriver}

taic\_driver通过,将硬件寄存器的读写映射到地址空间。通过MMIO的方式访问硬件寄存器。taic\_driver提供了对硬件队列的读写接口,用户线程可以通过该接口直接操作硬件队列。taic\_driver还提供了对能力表的读写接口,用户线程可以通过该接口直接操作能力表。

taic\_driver中主要接口如表\ref{tab:taicdriver}所示。

\begin{table}[htbp]
\centering
\begin{tabular}{|l|p{10cm}|}
\hline
\textbf{函数名} & \textbf{说明} \\
\hline
\texttt{alloc\_lq} & 分配一个就绪队列,根据 OS ID 和进程 ID 从硬件申请资源。 \\
\hline
\texttt{task\_enqueue} & 将任务句柄写入就绪队列,向硬件发起任务入队请求。 \\
\hline
\texttt{task\_dequeue} & 读取队列中的任务句柄,如果为 0 表示队列为空,否则返回有效任务句柄。 \\
\hline
\texttt{register\_sender} & 将本就绪队列注册为某进程的发送方。 \\
\hline
\texttt{cancel\_sender} & 撤销先前注册的发送方信息。 \\
\hline
\texttt{register\_receiver} & 注册接收端信息与处理句柄。 \\
\hline
\texttt{send\_intr} & 基于已注册的信息,向接收方发送中断 \\
\hline
\texttt{register\_extintr} & 注册外部中断处理句柄。 \\
\hline
\end{tabular}
\caption{taic\_driver 中 \texttt{LocalQueue} 接口列表}


\label{tab:taicdriver}
\end{table}

其中,alloc\_lq方法基于TAIC对象,用于分配一个就绪队列(Local\_Queue),其他接口均基于Local\_Queue对象调用。

alloc\_lq方法需提供OS\ ID和进程ID作为参数,OS\ ID用于标识操作系统实例,进程ID用于标识进程。如果当前进程中已有一个队列,该方法会申请另一套硬件资源并进行分配两个队列属于同一进程。其中,0号进程ID为内核进程保留使用。

register\_sender方法用于将本就绪队列注册为某进程的发送方。注册完成后,可通过该队列向其他队列发送中断信号。

register\_receiver方法用于注册接收端信息与处理句柄。该句柄的低二位作为配置位使用,分别表述该次注册是否可持续使用和该次注册是否抢占先前的注册行为。注册完成后,本队列收到中断信号时会将该句柄加入到就绪队列中。






\section{本章小结}

本章围绕本设计的相关工作展开论述,包括Rust并发模型、ReL4系统设计及TAIC任务感知中断控制器。Rust语言的异步编程基于async模型,采用协程和事件驱动,通过async/await语法实现高效异步编程,围绕Executor和Reactor设计异步运行时。ReL4是基于Rust的异步微内核,旨在针对seL4的IPC机制进行优化。ReL4通过对通知机制进行重构,引入用户态中断,并基于协程调度实现了异步运行时,实现了高性能的异步进程间通信。TAIC是一种任务感知中断控制器,通过创新性地将任务状态管理与调度逻辑集成于硬件,实现了中断处理的动态调整,提高系统响应速度和资源利用率。本章对其驱动设计与主要接口进行了介绍。